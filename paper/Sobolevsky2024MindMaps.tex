\documentclass[12pt]{article}
\usepackage[]{cite}
\usepackage{cmap}
\usepackage[T2A]{fontenc}
\usepackage[utf8]{inputenc}
\usepackage[english, russian]{babel}
\usepackage{amsmath, amsfonts,amssymb}
\usepackage{graphicx, epsfig}
\usepackage{subfig}
\usepackage{color}
\usepackage{hyperref}


% \newcommand\argmin{\mathop{\arg\min}}
% \newcommand{\T}{^{\text{\tiny\sffamily\upshape\mdseries T}}}
% \newcommand{\hchi}{\hat{\boldsymbol{\chi}}}
% \newcommand{\hphi}{\hat{\boldsymbol{\varphi}}}
% \newcommand{\bchi}{\boldsymbol{\chi}}
% \newcommand{\A}{\mathbf{A}}
% \newcommand{\bb}{\mathbf{b}}
% \newcommand{\B}{\mathcal{B}}
% \newcommand{\W}{\mathbf{W}}
% \newcommand{\E}{\mathbf{E}}
\newcommand{\x}{\mathbf{x}}
\newcommand{\y}{\mathbf{y}}
\newcommand{\Y}{\mathbf{Y}}
\newcommand{\X}{\mathbf{X}}
\newcommand{\D}{\mathcal{D}}
\newcommand{\s}{\mathbf{s}}
% \newcommand{\Z}{\mathbf{Z}}
% \newcommand{\hx}{\hat{x}}
% \newcommand{\hX}{\hat{\X}}
% \newcommand{\hy}{\hat{y}}
\newcommand{\M}{\mathcal{M}}
\newcommand{\I}{\mathcal{I}}
\newcommand{\Q}{\mathcal{Q}}
\renewcommand{\S}{\mathcal{S}}
% \newcommand{\N}{\mathcal{N}}
\newcommand{\R}{\mathbb{R}}
% \newcommand{\p}{p(\cdot)}
% \newcommand{\cc}{\mathbf{c}}
% \newcommand{\m}{\mathbf{m}}
% \newcommand{\bt}{\mathbf{t}}
% \newcommand{\e}{\mathbf{e}}
% \newcommand{\h}{\mathbf{h}}
% \newcommand{\q}{q(\cdot)}
% \newcommand{\uu}{\mathbf{u}}
% \newcommand{\vv}{\mathbf{v}}
% \newcommand{\dd}{\partial}

\renewcommand{\baselinestretch}{1}


\newtheorem{Th}{Теорема}
\newtheorem{Def}{Определение}
\newenvironment{Proof} % имя окружения
    {\par\noindent{\bf Доказательство.}} % команды для \begin
    {\hfill$\scriptstyle\blacksquare$} % команды для \end
\newtheorem{Assumption}{Предположение}
\newtheorem{Corollary}{Следствие}

\textheight=24cm % высота текста
\textwidth=16cm % ширина текста
\oddsidemargin=0pt % отступ от левого края
\topmargin=-1.5cm % отступ от верхнего края
\parindent=24pt % абзацный отступ
\parskip=0pt % интервал между абзацами
\tolerance=2000 % терпимость к "жидким" строкам
\flushbottom % выравнивание высоты страниц

%\graphicspath{ {fig/} }



\begin{document}

\thispagestyle{empty}
\begin{center}
    \sc
        Министерство образования и науки Российской Федерации\\
        Московский физико-технический институт
        {\rm(государственный университет)}\\
        Физтех-школа прикладной математики и информатики\\
        Кафедра <<Интеллектуальные системы>>\\
        Направление <<Интеллектуальный анализ данных>> \\[30mm]
    \rm\large
        Соболевский Федор Александрович\\
        Б05-111\\[10mm]
    \bf\Large
	Применение больших языковых моделей 
        для иерархической суммаризации
        текстов научных публикаций \\[10mm]
    \rm\normalsize
    \sc
        Выпускная квалификационная работа бакалавра\\[10mm]
\end{center}
\hfill\parbox{90mm}{
    \begin{flushleft}
    \bf
        Научный руководитель:\\
    \rm
        д.~ф.-м.~н. Воронцов Константин Вячеславович\\[5cm]
    \end{flushleft}
}
\begin{center}
    Москва\\
    2024
\end{center}


\newpage
\tableofcontents
\newpage

\begin{abstract}
В век экспоненциального роста количества доступной информации в мире особенно актуальной становится задача структурирования и систематизации научных знаний, а также повышения их доступности. Иерархическая организация основных идей и результатов в научных публикациях может позволить ускорить процесс получения читателем знаний и позволить ему двигаться при изучении темы от главного к деталям. Одним из видов структурированного представления текста являются интеллект-карты на основе предложений из текста. Поскольку человеческая обработка больших коллекций текстовых документов, особенно научных, занимает много времени и ресурсов, для решения задачи иерархической суммаризации необходимо разрабатывать автоматические методы, по качеству не уступающие ручной обработке. 

Перспективным инструментом решения данной задачи являются большие языковые модели. В данной работе исследуется способность больших языковых моделей строить иерархические представления текстов научных публикаций на примере интеллект-карт на основе предложений. Поскольку для задачи автоматической иерархической суммаризации научных текстов на данный момент не существует достаточного количества обучающих данных и метрик для разностороннего и полного оценивания качества генерации, предварительно проводится работа по созданию новой коллекции иерархических сводок научных статей для обучения и тестирования моделей иерархической суммаризации и предлагаются новые способы оценивания результатов выполнения данной задачи.

  \bigskip
  \textbf{Ключевые слова}: \emph{большие языковые модели, иерархическая суммаризация, интеллект-карты
  }
\end{abstract}

\newpage

%%%%%%%%%%%%%%%%%%%%%%%%%%%%%%%%%%%%%%%%%%%%%%%%%%%%%%%%%%%%%%%%%%%%%%%%%%%%%
\section*{Введение}
\addcontentsline{toc}{section}{\protect\numberline{}Введение}

% \paragraph{Актуальность темы.} 
% TO-DO

% - большие объемы информации, дезинформации и необходимость структурирования её}

% - необходимость более эффективного распространения новых научных знаний
% }

% - эффективность интеллект-карт из предложений как инструмента эффективного получения новых знаний и при этом практическое отсутствие выборок и информативных метрик для задачи иерархической суммаризации.

% - быстрое развитие больших языковых моделей, их SOAT результаты в сфере суммаризации и нераскрытый потенциал для задачи иерархической суммаризации.

\paragraph{Цели работы.}
\begin{itemize}
    \item Формализовать задачу автоматического построения интеллект-карт по научным публикациям и предложить метрики оценивания таких карт, достаточно хорошо отражающие реальное качество генерации по структуре, связности и достоверности;
    \item Разработать и применить методику построения интеллект-карт по научным текстам экспертами с целью формирования выборки - золотого стандарта для обучения и валидации моделей автоматической иерархической суммаризации, а также с целью выработки четких требований и целей для автоматической генерации;
    \item Применить большие языковые модели (БЯМ) для генерации интеллект-карт по текстам научных статей и определить оптимальные запросы, позволяющие максимизировать качество генерации для каждой выбранной БЯМ;
    \item Проанализировать свойства генерируемых с помощью БЯМ карт, их достоинства и недостатки самих по себе и в сравнении со стандартами и определить границы применимости современных БЯМ для иерархической суммаризации научной литературы. 
\end{itemize}


% \paragraph{Методы исследования.}
% TO-DO


% \paragraph{Основные положения, выносимые на защиту.}
% TO-DO


% \paragraph{Научная новизна.}
% TO-DO

% to my best knowledge, на данный момент нет ни адекватных критериев для автоматического оценивания карт знаний по научным статьям, ни выборки для данной задачи


% \paragraph{Теоретическая значимость.}
% TO-DO

% - формализация задачи гибридной иерархической суммаризации в виде интеллект-карт из предложений

% - разработка автоматических метрик оценки качества иерархических представлений текста по собственной структуре карты и в сравнении с референсом


% \paragraph{Практическая значимость.}
% TO-DO

% - разработка метода автоматической генерации представлений научных статей для их эффективного изучения с нужной степенью углубления в детали

% - датасет для задачи генерации интеллект-карт из предложений по научным текстам

% - (было бы неплохо?) фреймворк для автоматического оценивания подобных интеллект-карт


% \paragraph{Степень достоверности и апробация работы.}
% TO-DO

%%%%%%%%%%%%%%%%%%%%%%%%%%%%%%%%%%%%%%%%%%%%%%%%%%%%%%%%%%%%%%%%%%%%%%%%%%%%%
\newpage
\section*{Обозначения и сокращения}
\addcontentsline{toc}{section}{\protect\numberline{}Обозначения}
\begin{itemize}
    \item \textit{БЯМ}~--- большая языковая модель (large language model);
    
    \item Под \textit{картами знаний} или \textit{интеллект-картами} в данной работе будут подразумеваться интеллект-карты на основе предложений (salient-sentence-based mind maps, SSM).

    \item \textit{KSM}~--- интеллект-карта на основе главных выдержек (key-snippet based mind map).

    \item \textit{ROUGE} - Recall-Oriented Understudy for Gisting Evaluation (статистическая метрика качества суммаризации).

    \item \textit{BLEU} - Bilingual Evaluation Understudy (статистическая метрика качества генерации текста, основное применение~--- оценка машинного перевода).

    \item \textit{ИАТ}~--- интеллектуальный анализ текста (text mining).
\end{itemize}

%%%%%%%%%%%%%%%%%%%%%%%%%%%%%%%%%%%%%%%%%%%%%%%%%%%%%%%%%%%%%%%%%%%%%%%%%%%%%
\newpage
\section{Постановка задачи}
\subsection{Задача иерархической суммаризации}
Пусть дан документ (или коллекция документов) $\D$~--- упорядоченный набор предложений, составленных из слов некоторого словаря $V$: 
$$
\D = \left(s_i\right)_{i=1}^{|\D|},\quad \text{где } \forall i=1,\dots |\D|\quad s_i = \left(w_{ij}\right)_{j=1}^{l_i}, \quad w_{ij}\in V,
$$
а также референсная карта $\M^* = (\S^*, E^*)$ и метрика качества $\I: (\M, \D, \M^*) \rightarrow \R$. Тогда требуется найти отображение $f^*: \D \rightarrow \M = (\S, E)$, максимизирующее данную метрику качества $\I$, где $\M$~--- древовидная \textbf{иерархическая карта} (\textit{интеллект-карта}, \textit{карта знаний}), $\S$~--- набор предложений, являющихся вершинами $\M$ и составленных из слов словаря $V$, $E\in \S^2$~--- направленные иерархические связи между предложениями из $\S$, то есть ребра направленного графа $\M$:
$$
f^* = \arg\max\limits_{f} \I(f(\D), \D, \M^*).
$$

\subsection{Задача разработки критериев оценки карт знаний}
Поскольку на данный момент нет четких критериев для автоматической оценки качества иерархической суммаризации в виде карт знаний, перед тестированием БЯМ необходимо разработать систему таких критериев. Пусть мы имеем сгенерированную моделью по документу $\D$ карту знаний $\M$ и карту-стандарт $\M^*$ по тому же документу, созданную экспертами. 
Требуется определить критерии $$\I_k: (\M, \D, \M^*) \rightarrow \R$$ для автоматического оценивания интеллект-карт, отражающие интересующие нас аспекты качества генерации иерархического представления $\M$ относительно исходного документа $\D$ и карты-стандарта $\M^*$. 
Мера качества критерия~--- коэффициент корреляции с экспертной метрикой $\I^*_k$ оценки соответствующего аспекта качества по некоторой выборке карт $\X$. 

Выделим пять основных реальных аспектов качества карты знаний:
\begin{itemize}
    \item \textit{соответствие цели}, для которой создаётся данное представление документа;
    \item \textit{полнота} карты относительно документа, то есть содержание в ней всей необходимой пользователю информации на любом требуемом уровне абстракции и отсутствие лишнего;
    \item \textit{непротиворечивость} иерархического представления как на уровне предложений, так и между ними;
    \item \textit{связность} и \textit{неизбыточность} карты как набора осмысленных предложений и их последовательностей (подразумевается, что любой путь из корня дерева произвольной длины представляет из себя связный, неизбыточный текст);
    \item \textit{логичность} связей между вершинами в карте: степень логической связи между соединенными ребрами предложениями и корректность иерархии с логической точки зрения. 
\end{itemize}

В предыдущей постановке задачи неизбежно возникает следующая проблема: количество экспертных карт и скорость их создания сильно ограничены, поэтому после отработки сравнительных критериев следует разработать способ оценивать карты без стандартов. 

Пусть мы имеем только документ $\D$ и сгенерированную по нему моделью карту знаний $\M$. Требуется определить собственные критерии качества карты
$$\I_k: (\M, \D) \rightarrow \R,$$ 
отражающие качество генерации иерархического представления $\M$ исходного документа $\D$ самого по себе, без сравнения с другими картами. Мерой качества автоматического критерия, помимо степени скоррелированности с экспертными оценками, может послужить также результат оценивания с помощью него экспертных карт $\M^*$, принимаемых за стандарт.

\subsection{Задача оптимизации промптинга БЯМ}
Основной инструмент оптимизации работы готовой БЯМ без её дополнительного дообучения~--- подбор оптимального текстового запроса для решения задачи (\textit{промптинг}). Поскольку на данный момент нет достаточно быстрых алгоритмов поиска по всему множеству возможных запросов, а полный перебор запросов для каждой модели занимает слишком большое время, отдельной задачей в нашей работе является оптимизация промптинга БЯМ для цели генерации интеллект-карт и в целом.

Пусть дана выборка документов $\X$, языковая модель $f$, карта-стандарт $\M^*$ для заданной цели создания карты и некоторая метрика качества $\I$. Пусть также задано множество возможных запросов $\Q$, таких что вывод модели для каждого $(\D, Q)\in\X\times\Q$ соответствует требуемому формату, и содержащих формулировку цели, \textit{общую для модели и экспертов} - создателей $\M^*$. Требуется найти оптимальный запрос $Q^*\in\Q$, такой что вывод модели при входе $(\D, Q^*)$ максимизирует метрику $\I$ по выборке $\X$:
$$
Q^* = \arg\max\limits_{Q\in\Q} \frac{1}{|\X|}\sum\limits_{\D\in\X} \I(f(\D, Q), \D, ).
$$
Для эффективного поиска оптимального запроса (\textit{промптинга}) требуется также задать полное и неизбыточное множество запросов $\Q$ и эффективную \textit{стратегию поиска оптимального запроса} в $\Q$.

%%%%%%%%%%%%%%%%%%%%%%%%%%%%%%%%%%%%%%%%%%%%%%%%%%%%%%%%%%%%%%%%%%%%%%%%%%%%%%
\newpage
\section{Обзор}
\subsection{Методы суммаризации}

\paragraph{Суммаризация текстов и БЯМ.} Задача суммаризации текста представляет из себя задачу получения краткого представления $\S$ документа (или коллекции документов) $\D$. Выделяют два основных вида суммаризации: \textit{экстрактивную} (extractive), использующую предложения исходного документа ($\S\subset\D$), и \textit{абстрактивную}, то есть генерацию новых предложений на основе исходного текста ($\S\not\subset\D$). Также отдельно упоминается так называемая \textit{гибридная суммаризация} (hybrid summarization), подразумевающая извлечение из документа важных предложений с последующим их преобразованием. 

Хотя задача суммаризации появилась в научной литературе ещё во второй половине XX века \cite{luhn1958automatic}, основной объем работ по суммаризации текстов был опубликован уже в XXI году, причем до начала бурного развития архитектур глубокого обучения основные методы построения сводок документов и их оценки были в большинстве своем экстрактивными, основанными на статистических приемах обработки текстов \cite{allahyari2017text}. Абстрактивная суммаризация начала активно развиваться с момента появления трансформерных архитектур и других архитектур глубокой машинной обработки текста \cite{el2021automatic}. 

Особенных успехов в области удалось добиться с появлением больших языковых моделей, которые стали основным инструментом ля суммаризации текстов, так как показали значения метрик качества, которых до этого момента не удавалось добиться \cite{jin2024comprehensive}. Более того, способности БЯМ к пониманию, обработке и генерации текста нашли свое применение не только для автоматической генерации сводок, но и для их же оценивания и корректировки \cite{wu2023large}. Сравнение результатов работы БЯМ и человека по классической (линейной) суммаризации текстов на данный момент позволяет утверждать, что для некоторых типов текста машинная суммаризация с помощью БЯМ уже достигла уровня человека \cite{pu2023summarization}. Авторы данной работы, однако, подчеркивают, что проблема оценки качества нейросетевой суммаризации и генерации текста в целом до сих пор остается открытой, поэтому нельзя утверждать о полной достаточности БЯМ для суммаризации. Применимость БЯМ для других видов суммаризации текстов все еще остается мало исследованной.

\paragraph{Иерархическая суммаризация.} Идея структурированной суммаризации текстов как более эффективного способа суммаризации больших документов появилась в научной литературе еще в конце 2000-х гг. \cite{yang2008hierarchical}, однако впервые она была формализована в работе \cite{christensen2014hierarchical}. В первоначальной постановке задача иерархической суммаризации текста представляет из себя задачу генерации \textit{иерархии из сводок} по исходной коллекции документов, в которой дочерние сводки раскрывают содержание элементов (например, предложений) более общих сводок. Метод, представленный в \cite{christensen2014hierarchical}, подразумевает иерархическую кластеризацию предложений текста с последующей суммаризацией каждого кластера. Целевая функция в \cite{christensen2014hierarchical} отражает значимость выделенных предложений, неизбыточность и связность (как внутри элементов иерархии, так и между ними) полученной иерархической сводки. Хотя авторам удалось показать, что данный подход к суммаризации новостных текстов более предпочтителен среди пользователей, чем классические подходы к суммаризации новостных текстов, этот подход не получил дальнейшего развития. Более применимыми стали подходы, основанные на генерации на основе текстов \textit{интеллект-карт}.

\paragraph{Интеллект-карты.} На сегодняшний день существует множество различных видов организации знаний в виде графовых структур: онтологии, карты концепций (concept maps) и другие, но в данной работе нас будут интересовать \textit{интеллект-карты}, или \textit{карты знаний}. В работах по автоматической генерации интеллект-карт выделяют два основных вида интеллект-карт: \textit{интеллект-карты на основе значимых предложений} (salient sentence-based mind maps, SSM) и \textit{интеллект-карты на основе главных выдержек} (key snippet-based mind maps, KSM). В данной работе нас будут интересовать именно SSM как форма иерархической организации \textit{фактов} (в роли которых будут выступать связные предложения), однако следует подчеркнуть, что практическая значимость интеллект-карт в образовании и других областях исследовалась больше на примере KSM. Также обратим внимание на то, что интеллект-карты при применении их человеком зачастую не ограничиваются лишь иерархиями из текста и могут содержать как более сложные структурные элементы, так и нетекстовые элементы (например, визуальные).

После введения в употребление термина <<mind map>> Тони Бьюзеном в 1974 году интеллект-карты были экстенсивно изучены как инструмент представления, обработки и систематизации знаний. Многочисленные исследования применения интеллект-карт в школьном и высшем образовании как для презентации информации ученикам/студентам, так и для систематизации полученных знаний им же показали, что такой способ представления информации может заметно улучшить качество восприятия, запоминания и систематизации знаний студентами, в том числе при изучении научной литературы \cite{guerrero2015mind}. Подробный современный обзор применения интеллект-карт в образовании в XXI веке можно найти, например, в \cite{mitra2023tradition}.

\paragraph{Автоматическая генерация интеллект-карт.} В последнее десятилетие появился ряд работ по автоматической суммаризации текстов в виде интеллект-карт разных видов при помощи методов машинного обучения. Стоит отметить, что до этого были работы по генерации карт знаний/онтологий по текстам методами ИАТ, но в этих работах фокус больше направлен на моделирование взаимосвязей между отдельными словами/понятиями в тексте, чем между предложениями/фактами, поэтому для нашего исследования данные работы неактуальны.

В работе \cite{wei2019revealing} был предложен метод интеллект-карт (как KSM, так и SSM) по текстам следующим способом: а) с помощью сравнения эмбеддингов предложений строится граф взаимосвязей между ними; б) по графу взаимосвязей строится интеллект-карта нужного вида. Данная идея нашла свое развитие в работе \cite{hu2021efficient}, в которой авторы предложили более эффективный способ превращения графа отношений между предложениями в интеллект-карту с помощью модуля дистилляции графа (graph refinement module). В работе \cite{zhang2024coreference} эта идея была усовершенствована применением вместо графа отношений между предложениями, строящегося по эмбеддингам предложений, графа соотнесенности (coreference graph, discourse graph), строящегося по принципу, описанному в работе \cite{xu2019discourse}. В результате в \cite{zhang2024coreference} были полученные самые высокие значения используемых метрик качества, поэтому мы можем использовать данную модель в качестве базовой для сравнения результатов, полученных с помощью БЯМ, и результатов, достижимых с помощью специализированных нейросетевых архитектур.

В недавнее время были начаты исследования способности БЯМ к генерации подобных интеллект-карт. В работе \cite{jain2024structsum} с помощью промптинга больших языковых моделей строятся так называемые \emph{StructSum}~--- структурированные сводки текстов для поиска конкретной информации, в частности, таблицы и интеллект-карты (KSM). Авторы применяют \textit{самокритики} модели (critics) для улучшения качества генерации, такие как запросы по оцениванию самой моделью различных аспектов качества сгенерированного ею же StructSum и генерация вопросно-ответных пар по исходному тексту для проверки возможности находить нужную информацию из текста в полученной карте. Хотя тот формат карт и решаемые ими задачи, что исследуется в работе \cite{jain2024structsum}, несколько отличается от рассматриваемого в нашей работе, данное исследование подкрепляет предположение о том, что при достаточно изящной стратегии промптинга БЯМ могут стать качественным решением поставленной нами задачи.

\subsection{Оценка качества суммаризации}
\paragraph{Статистические критерии.} Общепринятым подходом к оцениванию суммаризации с начала развития методов решения данной задачи является использование статистических критериев качества. Самые часто используемые из них, семейство метрик ROUGE \cite{lin2004rouge}, основаны на количестве совпадающих текстовых единиц, таких как n-граммы, последовательности слов и пары слов, между сгенерированной сводкой и экспертным резюме. Другие подобные метрики качества~--- BLEU \cite{papineni2002bleu}, METEOR \cite{banerjee2005meteor}, MoverScore \cite{zhao2019moverscore} и другие~--- схожи с ROUGE по принципу работы в том смысле, что они оценивают сходство стандартной и сгенерированной сводок на уровне слов, словосочетаний, n-грамм и других небольших семантических единиц. 

Основной проблемой вышеперечисленных критериев является низкая репрезентативность статистического подхода в задаче оценки осмысленности, фактичности, связности и других более тонких аспектов сгенерированных сводок. Например, в работе \cite{fabbri2021summeval} по результатам масштабного сравнительного исследования автоматических критериев качества и экспертных оценок искусственно сгенерированных сводок новостных текстов был сделан вывод о том, что экспертные оценки некоторых аспектов реального качества суммаризации, какие как связность и актуальность сводки, достаточно низко коррелируют со значениями автоматических метрик, что указывает на серьезную проблему с автоматическим оцениванием генерации текста статистическими критериями. Это указывает на необходимость использования более сложных критериев качества, учитывающих смысловую структуру исходного текста и его стандартных и искусственно сгенерированных сводок.

\paragraph{Критерии, основанные на БЯМ.}Другим подходом к оцениванию качества суммаризации, ставшим довольно распространенным в последние пять лет, является оценивание суммаризации с помощью БЯМ. Из метрик, основанных на таком подходе, можно выделить BertScore \cite{zhang2019bertscore}, Shepherd \cite{wang2023shepherd} и Booookscore \cite{chang2023booookscore}. Несколько недавних работ также исследуют способность современных моделей по типу GPT оценивать качество суммаризации и искать ошибки в сгенерированных текстах. На данный момент такие методы также не являются полноценным решением проблемы оценивания качества суммаризации в силу неидеальности самих моделей, но они показывают многообещающие результаты. Более подробный обзор этих и других методов оценивания суммаризации с помощью БЯМ можно найти в \cite{jin2024comprehensive}. 

Отсутствие достаточно информативных метрик качества суммаризации остается основной проблемой в области суммаризации на сегодня. Во многих современных работах по автоматической суммаризации до сих пор используются простые статистические метрики по типу ROUGE и BLEU, причем зачастую смысл этих метрик не раскрывается, что делает сложным оценку реального качества генерируемых сводок. Хотя некоторыми исследователями были предприняты попытки систематического переосмысления оценивания качества суммаризации \cite{fabbri2021summeval}, \cite{zhang2024benchmarking}, на сегодняшний день задача разработки достаточных метрик для полного, разностороннего автоматического оценивания качества суммаризации остается нерешенной.

%%%%%%%%%%%%%%%%%%%%%%%%%%%%%%%%%%%%%%%%%%%%%%%%%%%%%%%%%%%%%%%%%%%%%%%%%%%%%%
\newpage
\section{Предлагаемый метод}

\paragraph{Сбор экспериментальных данных.} Для проверки работы БЯМ над задачей генерации интеллект-карт по научным публикациям необходима выборка таких карт, созданных людьми с достаточными компетенциями в данной задаче. Для этого и для отработки методики и целей построения иерархических сводок по научным документам первоначальна необходимо проведение самостоятельной работы по построению иерархических сводок научных статей. Предполагается создание набора первоначальных карт, в ходе построения которых, во-первых, должны быть установлены принципы, которыми следует руководствоваться при создании карты знаний, и, во-вторых, должны быть определены цели создания таких карт. Сформулируем несколько примерных целей создания карты знаний по научной публикации~--- получение минимальных знаний, необходимых для:
\begin{itemize}
\item воспроизведения результатов авторов статьи студентом или младшим научным сотрудником, разбирающимся в предметной области на базовом уровне;
\item выделения наиболее важного, нового, значимого результата для его популяризации или включения в образовательный курс по предметной области данного исследования;
\item выделения основного результата для упоминания в научном обзоре по предметной области;
\item подготовки пересказа статьи на научном семинаре, максимально близкого к тому, что могли бы рассказать сами авторы, желая донести свои результаты до профессионального сообщества в своей предметной области. 
\end{itemize}
Вполне естественно, что данными целями спектр применимости карт знаний не ограничивается, но для определенности в исследовании мы зафиксируем именно их.

\paragraph{Агрегация экспертных мнений.} Так как мнения экспертов по поводу оптимальных способов построения карт знаний могут розниться, эти мнения нужно агрегировать. Мы предлагаем следующую методику объединения экспертных усилий для совместного создания интеллект-карт по научным текстам: проведение научных семинаров в формате обсуждения интеллект-карт по статьям с последующим построением общей карты. Пусть мы собрали $N$ экспертов и задали $K$ целей создания интеллект-карты, тогда на выходе мы имеем по каждой обработанной статье $N+К$ карт, $K$ из которых выбираются нами в качестве стандартных для дальнейших исследований.

\paragraph{Отбор критериев качества.} Вышеупомянутые научные семинары можно использовать не только с целью создания стандартных интеллект-карт, но и для сбора экспертных оценок аспектов качества человеческих и сгенерированных искусственно интеллект-карт по рассматриваемым статьям. Это позволит вместе с созданием выборки собрать также данные для корреляционного анализа экспертных оценок и значений возможных критериев качества интеллект-карт, необходимого для последующего выбора критериев для оценивания машинной генерации иерархических карт автоматически.

%%%%%%%%%%%%%%%%%%%%%%%%%%%%%%%%%%%%%%%%%%%%%%%%%%%%%%%%%%%%%%%%%%%%%%%%%%%%%%
% \newpage
% \section{Вычислительный эксперимент}
% \subsection{Постановка эксперимента}

% \subsection{Экспериментальные данные}

% \subsection{Результаты}
        
%%%%%%%%%%%%%%%%%%%%%%%%%%%%%%%%%%%%%%%%%%%%%%%%%%%%%%%%%%%%%%%%%%%%%%%%%%%%%%
% \newpage
% \section*{Заключение}
% \addcontentsline{toc}{section}{\protect\numberline{}Заключение}

%%%%%%%%%%%%%%%%%%%%%%%%%%%%%%%%%%%%%%%%%%%%%%%%%%%%%%%%%%%%%%%%%%%%%%%%%
\newpage
\addcontentsline{toc}{section}{\protect\numberline{}Список литературы}
\bibliographystyle{ugost2008}
\bibliography{citations}

\end{document}

